\documentclass[slides]{pgnotes}

\title{Ansible Windows}

\begin{document}

\maketitle

\tableofcontents

\section{Lab preparation}

I recommend that you do Task 1 (VM setup) now.


\section{Common use cases}

From ansible guide: 

\begin{enumerate}

\item \textbf{Installing software}

\item \textbf{Installing updates}

\item \textbf{User and group management}

\item \textbf{Command execution}

\item \textbf{File transfer}

\item \textbf{Scheduled task execution}

\item \textbf{Service management}

\end{enumerate}

\section{Complications}

Although Ansible is somewhat \textit{cross-platform} there are some \textbf{specific issues} with Windows:

\begin{enumerate}

\item \textbf{Control node} must be running on \textbf{Linux}.

\begin{itemize}
\item Some experiemental support for Windows control nodes but in general it won't work.
\end{itemize}

\item \textbf{Different modules} for same tasks

\begin{itemize}
\item Things like user account creation that in theory are the same on both need different modules on Windows.
\item Playbooks aren't directly portable across Windows / Linux.
\end{itemize}

\item \textbf{WinRM} must be setup at a basic level.

\begin{itemize}
\item Ansible uses WinRM on Windows (rather than SSH) as transport.
\item WinRM can be confusing to setup securely!
\end{itemize}

\end{document}

